\documentclass[10pt, letterpaper]{tabcv}

%%%% XeLaTeX settings - Use these to set document fonts %%%
\usepackage{fontspec,xltxtra,xunicode}
\defaultfontfeatures{Mapping=tex-text}
%\setromanfont[Mapping=tex-text]{Adobe Garamond Pro}

% packages to make header
\usepackage{lettrine}
\usepackage{graphicx} % for resizebox

%%%%%%%%%%%%%%%%%%%%%%%%%%%%%%%%%%%%%%%%%%%%%%%%%%%%%%%%%%%%%%%%%%
%%%%%%  To highlight author name in publications sections   %%%%%%
%%%%%%%%%%%%%%%%%%%%%%%%%%%%%%%%%%%%%%%%%%%%%%%%%%%%%%%%%%%%%%%%%%
\bibliographystyle{myPlain}
% otherwise try, e.g. \bibliographystyle{plain}

% \myname{*} will appear bold in \pubsin{}
\def\myname{Katherine Renwick}%

% you could change \textbf to \emph or any other command
 \newcommand{\FormatName}[1]{%
   \edef\name{#1}%
   \ifx\name\myname
     \textbf{#1}%
   \else
        #1%
   \fi
 }
%%%%%%%%%%%%%%%%%%%%%%%%%%%%%%%%%%%%%%%%%%%%%%%%%%%%%%%%%%%%%%%%%%
%%%%%%%%%%%%%%%%%%%%%%%%%%%%%%%%%%%%%%%%%%%%%%%%%%%%%%%%%%%%%%%%%%


\begin{document}

% if there are publications to be added
% comment out otherwise
%\nobibliography{BottsPubs}

%%%%%%% Title/Heading/Name %%%%%%%%%%%%
% You're on your own here... 

\begin{flushright}
\textcolor{darkBlue}{ \Huge
\lettrine[lines=2,findent=-1pt, loversize = -0.42, lraise=0.6]{K}{atherine M. Renwick}}\\[-1.5pt]
\textcolor{Gray}{\resizebox{0.54\linewidth}{!}{Postdoctoral Researcher, Montana State University }}\\[-1.5pt]
\textcolor{Gray}{\resizebox{0.54\linewidth}{!}{607-229-3179 $\cdot$  katie.renwick@gmail.com $\cdot$ katierenwick.wordpress.com}}
\end{flushright}

%%%%%%%%%%%%%%%%%%%%%%%%%%%%%%%%%%%
%%%%%%%%%%%%%%%%%%%%%%%%%%%%%%%%%%%

\begin{cvblock}{Education}
    \cvitem{2015}{ {\bf Ph.D.} Colorado State University, Fort Collins, CO}
    \cvitem{}{Graduate Degree Program in Ecology}
    \cvitem{}{Graduate Teaching Certificate}

    \\[-6pt]

    \cvitem{2007}{ {\bf B.A.} Colby College, Waterville, ME}
    \cvitem{}{Double Major: Environmental Studies and Classics}
    \cvitem{}{\em summa cum laude}
    %\cvitem{M.S. Thesis}{\em A low-frequency investigation of coupled volume systems using finite-difference time-domain methods.}

    \\[-6pt]

\end{cvblock}


\begin{cvblock}{Professional Experience}
    \cvitem{2015--current}{{\bf Montana State University}, Bozeman, MT}
	    \cvitem{}{{\em Postdoctoral Researcher}, Dept. of Ecology}
    \cvitem{2015}{{\bf Colorado State University}, Fort Collins, CO}
	    \cvitem{}{{\em Instructor}, School of Global Environmental Sustainability}
    \cvitem{2010--2015}{{\bf Colorado State University}, Fort Collins, CO}
	    \cvitem{}{{\em Research Assistant}, Dept. of Ecosystem Science and Sustainability}
    \cvitem{2009--2010}{{\bf Cornell University}, Ithaca, NY}
	    \cvitem{}{{\em Research Assistant}, Boyce Thompson Institute for Plant Research}
	\cvitem{2009}{{\bf Cary Institute of Ecosystem Studies}, Millbrook, NY}
	    \cvitem{}{\em Project Assistant}
	\cvitem{2007--2008}{{\bf Medomak Valley Land Trust}, Waldoboro, ME}
	    \cvitem{}{\em Outreach and Stewardship Specialist}
	\cvitem{2007}{{\bf Acadia National Park}, Bar Harbor, ME}
	    \cvitem{}{\em Biological Science Technician}
	\cvitem{2006}{{\bf Colby College}, Waterville, ME}
	    \cvitem{}{{\em Undergraduate Research Assistant}, Dept. of Environmental Studies}
\end{cvblock}

\begin{cvblock}{Fellowships and Awards}
	\cvitem{2014}{{NASA-MSU Professional Enhancement Award}, Michigan State University}
	\cvitem{2013--2014}{{Sustainability Leadership Fellow}, School of Global Environmental Sustainability, CSU}
	\cvitem{2013}{{Honorable Mention, Best Student Presentation}, US-IALE}
	\cvitem{2012}{{Rocky Mountain Research Fellowship}, Rocky Mountain Conservancy}
	\cvitem{2011}{{Research Ambassador Fellow}, NSF and Evergreen State College}
	\cvitem{2011, 2012}{{Honorable Mention, NSF Graduate Research Fellowship Program}}
	%\cvitem{2011--2014}{{Travel Awards, 5 total from CSU, ESA, and NCEAS}}
	\cvitem{2010}{{Program of Research and Scholarly Excellence Fellowship}, Colorado State University}
	\cvitem{2010}{{University Graduate Fellowship}, Colorado State University}
\end{cvblock}

\begin{cvblock}{Teaching and Mentoring Experience}
	\cvitem{2016}{{Co-Instructor}, Advances in Ecological Modeling, Montana State University}
	\cvitem{2015}{{Instructor}, Global Environmental Sustainability, Colorado State University}
	\cvitem{2014}{{Co-Instructor}, Multivariate Statistics, Colorado State University}
	\cvitem{2014}{{Teaching Assistant}, Ecosystem Ecology, Colorado State University}
	\cvitem{2013, 2014}{{Research Mentor}, Undergraduate Research Program, Colorado State University}
	\cvitem{2012, 2013}{{Field Skills Instructor}, Undergraduate Research Program, Colorado State University}
	\cvitem{2012}{{Teaching Assistant}, Forest Ecology, Colorado State University}
	\cvitem{2004-2007}{{Writing Instructor and Tutor}, Colby College}
\end{cvblock}

\begin{cvblock}{Publications and Reports}
	\cvitem{2016}{{\bf Renwick, K.M.}, Rocca, M.E., and T.J. Stohlgren. Biotic disturbance facilitates range shift at the trailing but not the leading edge of lodgepole pine’s altitudinal distribution. {\em Journal of Vegetation Science}, 27(4):780-788.}
	\cvitem{2015}{{\bf Renwick, K.M.} and M.E. Rocca. Temporal context affects the observed rate of climate-driven range shifts in tree species. {\em Global Ecology and Biogeography}, 24:44-51.}
	\cvitem{2012}{{\bf Renwick, K.M.}, and M.E. Rocca. Impacts of climate change and mountain pine beetle along forest ecotones. {\em Report to Rocky Mountain National Park}.}
	\cvitem{2011}{{\bf Renwick, K.M.}, and M.E. Rocca. Aspen recruitment in mixed aspen-lodgepole stands: will the mountain pine beetle outbreak benefit aspen? {\em Report to Rocky Mountain National Park}.}
	\cvitem{In review}{{\bf Renwick, K.M.}, and M.E. Rocca. The importance of non-climatic constraints on the distribution and migration potential of Rocky Mountain tree species. {\em Journal of Ecology}.}
	\cvitem{In review}{Druckenbrod, D., Martin-Benito, D., Orwid, D., Pederson, N., Poulter, B., {\bf Renwick, K.M.}, and H. Shugart. Redefining temperate forest response to climate and disturbance in eastern North America: new insights at the mesoscale. {\em Ecography}.}
\end{cvblock}

%\begin{cvblock}{Publications and Reports}
%	\pubsin{2013}{Botts2013c,Botts2013b,Botts2013a,Henderson2013,Botts2013,Jasa2013,Luizard2013}
%	\pubsin{2012}{Botts2012d,Botts2012c,Henderson2012a}
	%\pubsin{2010}{Xiang2010,Botts2010a}
%	\pubsin{2009}{Tijs2009}
%\end{cvblock}

\begin{cvblock}{Selected Presentations}
	\cvitem{2016}{{\bf Renwick, K.M.}, Kleinhesselink, A.K., Curtis, C., Schlaepfer, D., Bradley, B., Poulter, B., and P. Adler. ``Forecasting changes in the distribution and abundance of sagebrush under climate change," Sagebrush Ecosystems Conservation Conference, Salt Lake City, UT}
	\cvitem{2015}{{\bf Renwick, K.M.} and M.E. Rocca. ``Incorporating non-climatic variables into species distribution models to improve forecasts of landscape-scale range shifts," Open Science Conference, Fort Collins, CO}
	\cvitem{2014}{{\bf Renwick, K.M.}, and M.E. Rocca. ``Climate change impacts at the range margins of Rocky Mountain tree species: interactions with disturbance and implications for future forests," Mountain Climate Research Conference, Midway, UT }
	\cvitem{2014}{{\bf Renwick, K.M.}, and M.E. Rocca. ``The relative importance of biotic variables in determining current and future tree species distributions," US-IALE Annual Symposium, Anchorage, AK}
	\cvitem{2013}{{\bf Renwick, K.M.}, and M.E. Rocca. ``Interactive effects of climate and disturbance on landscape-scale range dynamics of Rocky Mountain tree species," US-IALE Annual Symposium, Austin, TX {\em *Awarded Honorable Mention in the graduate student presentation competition}}
	\cvitem{2013}{{\bf Renwick, K.M.}, and M.E. Rocca. ``Are Rocky Mountain tree species responding to climate change?" Front Range Student Ecology Symposium, Fort Collins, CO {\em*Awarded 3rd place in the graduate student presentation competition}}
	\cvitem{2012}{{\bf Renwick, K.M.}, Nelson, K. and M.E. Rocca. ``Will the mountain pine beetle outbreak benefit aspen?" Rocky Mountain National Park Research Conference, Estes Park, CO}
	\cvitem{2012}{{\bf Renwick, K.M.} and M.E. Rocca. ``Recruitment patterns in mixed aspen-lodgepole stands: will the mountain pine beetle outbreak benefit aspen?" Front Range Student Ecology Symposium, Fort Collins, CO}
	\cvitem{2007}{{\bf Renwick, K.M.}, Sadanowicz, A., Dubuque, G., Becker, R., Thompson, C. and D. Garneau. ``Small mammal assemblages under urban influence in central Maine," Northeast Regional Sigma Xi Conference, Ithaca, NY}
	\cvitem{2007}{{\bf Renwick, K.M.}, Harrison, D., Adelphio, A., Firmage, D., Garneau, D. and K. Ness. ``Analysis of land-use patterns in the Long Pond watershed and effects on water quality," Maine Water Conference, Augusta, ME {\em*Awarded 1st place in the undergraduate student presentation competition}}
\end{cvblock}

\begin{cvblock}{Outreach and Service}
    \cvitem{Reviewer}{Global Ecology and Biogeography, Journal of Plant Ecology, Ecological Modeling}
    \cvitem{}{}
    \cvitem{Professional}{Ecological Society of America, Early Career Ecologists Section}
    \cvitem{Society}{International Association for Landscape Ecologists}
    \cvitem{Membership}{Association for Women in Science}
    \cvitem{}{}
    \cvitem{University}{Editorial Committee, Early Career Climate Forum}
    \cvitem{Service}{Newsletter Editor, Graduate Degree Program in Ecology (2013--2014)}
    \cvitem{}{Outreach Chair, Front Range Student Ecology Symposium (2013--2014)}
    \cvitem{}{Graduate Student Rep., Dept. of Ecosystem Science and Sustainability (2012--2013)}
    \cvitem{}{President, Front Range Student Ecology Symposium (2011--2012)}
    \cvitem{}{Fundraising Chair, Front Range Student Ecology Symposium (2010--2011)}
    \cvitem{}{Student Rep., Faculty Search Commitee, Colorado State University (2011)}
    \cvitem{}{President, Colby College Environmental Coalition (2005-2007)}
\end{cvblock}

\begin{cvblock}{Skills and Interests}
	\cvitem{Research}{Plant community dynamics, global change impacts, forest and rangeland management, ecosystem modeling, applied statistics}
	\cvitem{Teaching}{general ecology, biology/biodiversity, forest ecology, botany, research methods, data analysis/statistics, environmental sustainability, science for non-majors}
	\cvitem{Programming}{R $\cdot$ Markdown $\cdot$ JAGS $\cdot$ Shell $\cdot$ CDO $\cdot$ \LaTeX $\cdot$ Git $\cdot$ basic C++}
	\cvitem{}{\em examples at https://github.com/krenwick}
\end{cvblock}
%\clearpage

\end{document}
